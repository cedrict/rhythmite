\documentclass[a4paper]{article}
\usepackage[cm]{fullpage}
\usepackage{xcolor}
\usepackage{url}
\usepackage{bm}
\usepackage{listings}
\usepackage{amsmath}
\usepackage{amsfonts}

\newcommand{\nn}{\nonumber}
\newcommand{\python}{\color{darkgray} \sffamily }
\lstset{ 
  language=Python,
  backgroundcolor=\color{white},   % choose the background color; you must add \usepackage{xcolor}; should come as last argument
  basicstyle=\footnotesize,        % the size of the fonts that are used for the code
  breakatwhitespace=false,         % sets if automatic breaks should only happen at whitespace
  breaklines=true,                 % sets automatic line breaking
  captionpos=b,                    % sets the caption-position to bottom
  frame=single,                    % adds a frame around the code
  keepspaces=true,                 % keeps spaces in text, useful for keeping indentation of code 
  keywordstyle=\color{blue},       % keyword style
}


\usepackage[
backend=biber,
style=alphabetic,
sorting=ynt
]{biblatex}

\addbibresource{manual.bib}

\title{Diagenetic Self-Organization: an exercise in replication}
\author{C. Summers \& C. Thieulot}


\begin{document}
\maketitle

%%%%%%%%%%%%%%%%%%%%%%%%%%%%%%%%%%%%%%%%%%%%%%%%%%%%%%%%%%%%%%%%%%%%%%%%%%%%%%%%%%%%%%%%%%%%%%%%%%%

\subsubsection*{Context}

In the spring of 2024 we were approached by Emilia with regards to an article she and 
her team wished to reproduce. Although recently published, this article did not include an open source code. The article in question is \cite{lheu18}.\\\\ 
Can we reproduce the results of the paper using only the equations provided?

%----------------------------
\subsubsection*{The PDEs of the physical problem}
The model describes the evolution of the concentrations of five variables as a set of 5 coupled nonlinear advection-diffusion-reaction PDEs in a 1D domain:
\begin{itemize}
\item Solids: two minerals which dissolve and (re)precipitate from/to the same
solutes. They are expressed as proportions of the solid phase, and must
accordingly be non-negative, and their sum must be smaller or equal to one.
$C_C$ is the proportion of calcite in the solid phase, and $C_A$ is the 
proportion of aragonite in the solid phase.

\item Solutes $\hat{c}_k$, where $k \in \{ Ca,CO_3\}$.
with $\hat{c}_{Ca}$: calcium ion concentration in pore water, and $\hat{c}_{CO_3}$: carbonate ion concentration in pore water.  As concentrations these cannot be negative but are otherwise unlimited.
\item Porosity $\phi$
\end{itemize}
The equations involving our five unknowns $C_A$, $C_C$, $\hat{c}_k$ and $\phi$ are as follows (in order to highlight how nonlinear and coupled these equations are, I have colored the five unknowns throughout):
\begin{eqnarray}
\frac{\partial {\color{teal} C_A}}{\partial t} 
&=& -U({\color{teal}\phi}) \frac{\partial {\color{teal}C_A}}{\partial x} 
- Da[(1-{\color{teal}C_A}){\color{teal}C_A}(\Omega_{DA}-\nu_1\Omega_{PA})+
\lambda {\color{teal}C_AC_C} (\Omega_{PC}-\nu_2\Omega_{DC})]
\nn\\
\frac{\partial {\color{teal}C_C}}{\partial t} 
&=& -U({\color{teal}\phi}) \frac{\partial {\color{teal}C_C}}{\partial x}  
+ Da[(1-{\color{teal}C_C}){\color{teal}C_C}(\Omega_{PC}-\nu_2\Omega_{DC})+
\lambda {\color{teal}C_AC_C} (\Omega_{DA}-\nu_1\Omega_{PA})]
\nn\\
\frac{\partial {\color{teal}\hat{c}_k}}{\partial t} 
&=& -W({\color{teal}\phi}) \frac{\partial {\color{teal}\hat{c}_k}}{\partial x}
+\frac{1}{\color{teal}\phi} \frac{\partial}{\partial x} 
\left( {\color{teal}\phi} d_k \frac{\partial {\color{teal}\hat{c}_k}}{\partial x} \right)
+Da \frac{1-{\color{teal}\phi}}{{\color{teal}\phi}}(\delta-{\color{teal}\hat{c}_k})
[{\color{teal}C_A}(\Omega_{DA}-\nu_1\Omega_{PA})-\lambda 
{\color{teal}C_C} (\Omega_{PC}-\nu_2 \Omega_{DC})  ]
\nn\\ 
\frac{\partial {\color{teal}\phi}}{\partial t} 
&=& -\frac{\partial}{\partial x} (W({\color{teal}\phi}) {\color{teal}\phi})
+d_{\phi} \frac{\partial^2 {\color{teal}\phi}}{\partial x^2} + Da (1-{\color{teal}\phi})
[{\color{teal}C_A}(\Omega_{DA}-\nu_1 \Omega_{PA})-\lambda 
{\color{teal}C_C} (\Omega_{PC}-\nu_2 \Omega_{DC})] \nn
\end{eqnarray}
$k=1$ corresponds to $\hat{c}_{Ca}$, and and $k=2$ corresponds to $\hat{c}_{C0_3}$.
We assume that the coordinate $x=0$ corresponds to the top of the domain
which is of length $L$. 
The PDEs above are actually dimensionless (the primes have been dropped), with 
\[
x'=x/X^\star, \quad t'=t/T^\star \quad \hat{c}_k'=\hat{c}_k/\sqrt{K_C},
\quad U=u/S, \quad W=w/S.
\]
The scaling factors are given by $X^* = D_{Ca,0} / V_s$ and $T^* = D_{Ca,0} / V_s^2 $, where $D_{Ca,0}$ is the initial calcium ion diffusion coefficient and $V_s$ is the sedimentation rate. The other constants in the equations are given by 
\[
Da=k_2 T^\star = \frac{k_2 D_{Ca}^0}{S^2},
\quad
\lambda=k_3/k_2,
\quad
\nu_1=k_1/k_2,
\quad
\nu_2=k_4/k_3,
\quad
d_k=D_k/D_{Ca}^0, 
\quad
d_\phi=D_\phi/D_{Ca}^0,
\quad
\delta=\frac{\rho_s}{\mu_A \sqrt{K_C}}.
\]
Da can be interpreted as a Damköhler number\footnote{\url{https://en.wikipedia.org/wiki/Damkohler_numbers}}. 


%......................................
\subsubsection*{Hydraulic conductivity}
The velocities in the above PDEs depend on the hydraulic conductivity, which in turn depends on the porosity through an empirically determined formula.  In this model it is given by
\[
K(\phi)
=\beta \frac{\phi^3}{(1-\phi)^2} F(\phi)
=\beta \frac{\phi^3}{(1-\phi)^2} \left[ 1-\exp\left( -\frac{10(1-\phi)}{\phi} \right) \right] 
\]
which translated into the following code snippet:
\begin{lstlisting}
def K(self,phi):
    return self.beta*(phi**3/(1-phi)**2)*(1-np.exp(-10*(1-phi)/phi))
\end{lstlisting}
The $F(\phi)$ expression is chosen such that in the limit as $\phi \rightarrow 1$ the hydraulic conductivity takes on the form specified in Equation 16 of \cite{lheu18}.  Based on this, it is interesting to consider the behaviour of $K$ at both limits:
\[
\lim_{\phi\rightarrow 0} K(\phi) = ?
\qquad
\lim_{\phi\rightarrow 1} K(\phi) = ?
\]
Let us recall the Taylor expansion of the exponential function in zero:
\[
\exp (x) \simeq 1 + x + \frac{x^2}{2} + \frac{x^3}{6} + \dots
\]
When $\phi \rightarrow 1$, then $(1-\phi)/\phi \rightarrow 0$ and then 
\[
1-\exp\left( -\frac{10(1-\phi)}{\phi} \right) \sim 1- (1 -\frac{10(1-\phi)}{\phi} ) = \frac{10(1-\phi)}{\phi}
\]
so that 
\[
\lim_{\phi\rightarrow 1} K(\phi) = \beta \frac{\phi^3}{(1-\phi)^2} \cdot \frac{10(1-\phi)}{\phi}
=10 \beta \frac{\phi^2}{1-\phi}
\]
which is Eq.~(16) of the paper.

Also, since When $\phi \rightarrow 0$, then $(1-\phi)/\phi \rightarrow +\infty$ and then
\[
\lim_{\phi \rightarrow 0} \left( 1-\exp\left( -\frac{10(1-\phi)}{\phi} \right) \right) = 1 
\]
and finally
\[
\lim_{\phi\rightarrow 0} K(\phi) = 0
\]

%............................
\subsubsection*{Velocities}
$U$ is the velocity of the solid minerals and $W$ is the velocity of the pore water.  These are given by
\begin{eqnarray}
U(\phi)&=& 1-\frac{K(\phi^0)}{S}(1-\phi^0)(\frac{\rho_s^0}{\rho_w}-1) 
+\frac{K(\phi)}{S}(1-\phi) (1-\phi^0)(\frac{\rho_s}{\rho_w}-1)    \\
W(\phi)&=& 1-\frac{K(\phi^0)}{S}(1-\phi^0)(\frac{\rho_s^0}{\rho_w}-1) 
-\frac{K(\phi)}{S} \frac{(1-\phi)^2}{\phi} (1-\phi^0)(\frac{\rho_s}{\rho_w}-1)   
\end{eqnarray}
These translate into
\begin{lstlisting}
def U(self, phi):
    u = 1 - ( 1 / self.sed_rate )*\
            ( self.K(self.phi_0) * ( 1 - self.phi_0 ) - self.K(phi) * ( 1 - phi ) )*\
            ( self.rho_s0 / self.rho_w - 1 )
    return u

def W(self,phi):
    w = 1-( 1 / self.sed_rate )*\
          ( self.K(self.phi_0) * ( 1 - self.phi_0 ) + self.K(phi)*(1-phi)**2/phi)*\
          ( self.rho_s0 / self.rho_w - 1 )
    return w
\end{lstlisting}
The $U$ velocity contains the term $K(\phi)(1-\phi)$
and the $W$ contains the term $K(\phi)(1-\phi)^2/\phi$.
We find
\begin{eqnarray}
\lim_{\phi\rightarrow 0} K(\phi)(1-\phi) &=& 0 \nn\\
\lim_{\phi\rightarrow 1} K(\phi)(1-\phi) &=& 10 \beta \frac{\phi^2}{1-\phi}(1-\phi) = 10 \beta\nn\\
\lim_{\phi\rightarrow 0} K(\phi)(1-\phi)^2/\phi 
&=& \beta  \frac{\phi^3}{(1-\phi)^2} (1-\phi)^2/\phi = \beta \phi^2 = 0 \nn\\
\lim_{\phi\rightarrow 1} K(\phi)(1-\phi)^2/\phi &=& 10 \beta \frac{\phi^2}{1-\phi} (1-\phi)^2/\phi = 0 \nn
\end{eqnarray}

%..............................................
\subsubsection*{Porosity diffusion coefficient}

$d_\phi$ is the dimensionless form, calculated as $d_\phi=D_\phi/D^0_{Ca}$.
In the article and in our calculations, it is set to be constant given by:

\begin{lstlisting}
self.d_phi = self.beta*(self.phi_init**3 / ( 1 - self.phi_init ) )*\ 
             (1 / ( self.b*self.g*self.rho_w*( self.phi_NR - self.phi_inf ) ) )*\ 
             (1 - np.exp( -10*( 1 - self.phi_init ) / self.phi_init) )*( 1 / self.D_Ca_0 )
\end{lstlisting}

%..............................................
\subsubsection*{$k$ components diffusion coefficient}

$D_k(\phi)$ the (tortuosity-corrected) diffusion
coefficient of component $k$.
In Eq.~6 of the paper we find 
\[
D_k = \frac{D_k^0}{1- \log \phi^2}
\]
where $D_k^0$ is the diffusion of component $k$ in seawater.

\begin{lstlisting}
def d_c_ca(self, phi):
        # scaled with D_Ca_0
        return 1.0/( 1 - 2*np.log(phi) )

def d_c_co(self, phi):
        # scaled with D_Ca_0
        return self.D_CO/self.D_Ca_0*(1 / ( 1 - 2*np.log(phi) ) )
\end{lstlisting}




%....................................
\subsubsection*{Saturation factors}

$\Omega_{PA}$ and $\Omega_{DA}$ are saturation factors for precipitation and dissolution of aragonite. $\Omega_{PC}$ and $\Omega_{DC}$ are saturation factors for precipitation and dissolution of calcite.  The oversaturation factor for Aragonite are given by
\[
\Omega_{PA}=\left( \frac{\hat{c}_{CA} \hat{c}_{CO3} K_C}{K_A} -1 \right)^{m'}
\]
where $m'$ is a reaction order (different from $m$ in general).  It is understood that this reaction is only effective when the system is oversaturated, i.e. $\hat{c}_{CA} \hat{c}_{CO3} K_C/K_A -1 >0$.  The undersaturation factor is then given by
\[
\Omega_{DA}= \left(1-\frac{\hat{c}_{CA} \hat{c}_{CO3} K_C}{K_A}  \right)^m \theta(x)
\]
where $K_A$ is the solubility of aragonite and $m$ is a reaction order. It is understood that this reaction occurs only when the system is undersaturated, i.e. $1-\hat{c}_{CA} \hat{c}_{CO3}/K_A >0$.  The factor $\theta(x)$ is a characteristic function that is zero when $x$ is outside the aragonite dissolution zone (ADZ) and one otherwise. The ADZ is characterized by the position of its top edge $x_d$ and its thickness $h_d$.\\\\
Similarly, the over and under saturation factors for calcite are 
\[
\Omega_{PC}=\left( \hat{c}_{CA} \hat{c}_{CO3} -1 \right)^{n'}
\]
where again this reaction is only effective when $\hat{c}_{CA} \hat{c}_{CO3} -1 >0$ and 
\[
\Omega_{DC}=\left( 1- \hat{c}_{CA} \hat{c}_{CO3}\right)^{n'}
\]
where the reaction is only active when $1- \hat{c}_{CA} \hat{c}_{CO3} >0$.\\\\
Note that there is a confusion about $m$ and $m'$ in eqs 27 and 45 of the paper, but in practice we set $m=m'$ and $n=n'$.\\\\
Looking at the PDEs we find that $\Omega_{DA}$ and $\Omega_{PA}$ always occur together so we define
\[
\Omega_A=\Omega_{DA}-\nu_1\Omega_{PA} 
\]
Likewise we define
\[
\Omega_C=\Omega_{PC}-\nu_2\Omega_{DC}
\]
leading to their implementation in the code as:
\begin{lstlisting}
def Omega_A(self, c_ca, c_co, x):
    sp = c_ca*c_co*self.K_C/self.K_A - 1
    Omega_PA = (max(0.0,sp))**self.m
    sa = 1 - c_ca*c_co*self.K_C/self.K_A
    Omega_DA = (max(0.0,sa))**self.m * heaviside(x,self.ADZ_bot,self.ADZ_top,self.x_scale)
    return Omega_DA - self.nu1*Omega_PA

def Omega_C(self, c_ca, c_co):
    sp = c_ca*c_co - 1
    Omega_PC = (max(0.0,sp))**self.n
    sa = 1 - c_ca*c_co
    Omega_DC = (max(0.0, sa))**self.n
    return Omega_PC - self.nu2*Omega_DC
\end{lstlisting}

%----------------------------------------------------------------------
\subsection*{Numerical methods}
We solve the coupled PDEs by the Method of Lines, i.e. the right hand side of the equations are discretised by means of the Finite Difference Method but will keep the time derivative as they are and then use an Initial Value Problem integrator, e.g. Runge-Kutta methods.\\\\
We have implemented two methods: a simple 1st order Euler method (with a user-chosen constant timestep $dt$), and the use of the {\python solve\_ivp} from {\python scipy}. %(see \stone~156,157). 
The code has been designed with modularity in mind so that many functions have been created, each carrying out a single task.  In addition, as the equations require many constant parameters, we created a class to contain all the functions and parameter variables.  This way the functions need only take the solution variables and the `self' keyword as arguments, rather than passing lengthy lists of parameters to each function.  Parameter variables are set in the constructor function `\_\_init\_\_(self)' then accessed within the function using `self.var\_name'.\\\\
We also applied jit (just in time) compiling to the entire class, using the Numba package\footnote{\url{https://numba.readthedocs.io/en/stable/}}.  This produces compiled binaries for python functions at the first time they are encountered in the run and then reuses these binaries for all subsequent function calls, resulting in dramatically improved performance.  For this code, speed-up varies between a factor of roughly 3-10, depending on the mode the code runs in.  The addition of jit to the class is relatively simple, the main addition is that the type of all parameter values must be explicitly specified and passed to the jit decorator as a list of 2-element tuples specifying the variable name and its type:
\begin{lstlisting}
from numba.experimental import jitclass

# for the jit-compiling, we need to specify the type of all parameters in the class
spec = [
    ('g', float64),
    ('K_C', float64),
    ('K_A', float64),
    ('ADZ_bot', float64),
    ('ADZ_top', float64),
    .
    ..
    ...
    ]
# This class contains the functions and parameters that are used to calculate
# the RHS of eqns 40-43 of L'Heureux (2018)
@jitclass(spec)
class LHeureux:
    
    # define all params as instance vars
    # this way we can easily modify them at the instance level
    # i.e. for parameter searches
    def __init__(self):
        # physical constants
        self.g = 9.81*100           # gravitational acceleration (cm/s^2)
        # model parameters
        self.K_C = 10**-6.37        # Calcite solubility (M^2)
        self.K_A = 10**-6.19        # Aragonite solubility (M^2)
        self.ADZ_top = 50           # top of the Aragonite dissolution zone (cm)
        self.ADZ_bot = 150
        .
        ..
        ...
\end{lstlisting}
For clarity we write the five equations as
\begin{eqnarray}
\frac{\partial C_A}{\partial t} &=& -U \frac{\partial C_A}{\partial x}  + {\cal R}_{\text Aragonite}
\nn\\
\frac{\partial C_C}{\partial t} &=& -U \frac{\partial C_C}{\partial x}  + {\cal R}_{\text Calcite} 
\nn\\
\frac{\partial \hat{c}_k}{\partial t} &=& -W \frac{\partial \hat{c}_k}{\partial x}
+\frac{1}{\phi} \frac{\partial}{\partial x} \left( \phi d_k \frac{\partial \hat{c}_k}{\partial x} \right)
+{\cal R}_{k}
\nn\\ 
\frac{\partial \phi}{\partial t} &=& -\frac{\partial}{\partial x} (W \phi) +{\cal R}_\phi
\end{eqnarray}
The ${\cal R}$ rhs terms are implemented via five functions:
\begin{lstlisting}
#Aragonite
def R_AR(self, AR, CA, c_ca, c_co, x):
    return - self.Da*( ( 1 - AR ) * AR * self.Omega_A(c_ca, c_co, x) +\
                        self.lamb * AR * CA * self.Omega_C(c_ca, c_co) )

#Calcite
def R_CA(self, AR, CA, c_ca, c_co, x):
    return self.Da*( self.lamb * ( 1 - CA ) * CA * self.Omega_C(c_ca, c_co) +\
                        AR * CA * self.Omega_A(c_ca, c_co, x) )

# Ca ions
def R_c_ca(self, AR, CA, c_ca, c_co, phi, x):
    return self.Da*( ( 1 - phi ) / phi ) * (self.delta - c_ca)*\
               ( AR * self.Omega_A(c_ca, c_co, x) - self.lamb * CA * self.Omega_C(c_ca, c_co) )

# CO3 ions
def R_c_co(self, AR, CA, c_ca, c_co, phi, x):
    return self.Da*( ( 1 - phi ) / phi ) * (self.delta - c_co)*\
               ( AR * self.Omega_A(c_ca, c_co, x) - self.lamb * CA * self.Omega_C(c_ca, c_co) )

# porosity
def R_phi(self, AR, CA, c_ca, c_co, phi, x):
    return self.Da*( 1 - phi )*\
               ( AR * self.Omega_A(c_ca, c_co, x) - self.lamb * CA * self.Omega_C(c_ca, c_co) )
\end{lstlisting}
Because sharp transitions and discontinuities are very often source of error in numerical modelling we added an option to replace the Heaviside function $\theta(x)$ (required within the {\python Omega\_A} function) by a smoothed version (the value of 500 controls the smoothness):
\begin{lstlisting}
def heaviside(x,xbot,xtop,xscale, smooth_switch):
    if (smooth_switch==False):
        if x < xbot/xscale and x > xtop/xscale:
          return 1.0
        else:
          return 0.0
    else:
        val=0.5*(1+np.tanh((x-xtop/xscale)*500)) *0.5*(1+np.tanh((xbot/xscale-x)*500))
        return val
\end{lstlisting}
The solution for all variables is stored in a single array, 5*nnx long, with the first nnx elements corresponding to the AR values, the next nnx giving CA and so on.  The full ordering of the variables in X is AR, CA, c\_ca, c\_co, phi.  This structure is designed for use with the ivp\_solve function, which expects a single array for the solution variable values.  ivp\_solve also expects a single function which calculates the RHS, so we also combine the RHS functions in a master function:
\begin{lstlisting}
def X_RHS(self, t, X, nnx, x, h):
        
dAR_dt   = self.RHS_AR(  X[0:nnx], X[nnx:2*nnx], X[2*nnx:3*nnx], X[3*nnx:4*nnx], X[4*nnx:5*nnx], x, h)
dCA_dt   = self.RHS_CA(  X[0:nnx], X[nnx:2*nnx], X[2*nnx:3*nnx], X[3*nnx:4*nnx], X[4*nnx:5*nnx], x, h)
dc_ca_dt = self.RHS_c_ca(X[0:nnx], X[nnx:2*nnx], X[2*nnx:3*nnx], X[3*nnx:4*nnx], X[4*nnx:5*nnx], x, h)
dc_co_dt = self.RHS_c_co(X[0:nnx], X[nnx:2*nnx], X[2*nnx:3*nnx], X[3*nnx:4*nnx], X[4*nnx:5*nnx], x, h)
dphi_dt  = self.RHS_phi( X[0:nnx], X[nnx:2*nnx], X[2*nnx:3*nnx], X[3*nnx:4*nnx], X[4*nnx:5*nnx], x, h)
        
return np.concatenate((dAR_dt, dCA_dt, dc_ca_dt, dc_co_dt, dphi_dt)) 
\end{lstlisting}


%----------------------------------------------------------------------
\subsection*{Initial values}

The initial conditions are chosen as spatially homogeneous constants:
\begin{eqnarray}
C_A(x,0)&=&C_{A,init} \\
C_C(x,0)&=&C_{C,init} \\
\hat{c}_k(x,0)&=&\hat{c}_{k,init} \\
\phi(x,0) &=& \phi_{init}
\end{eqnarray}
for $x\ne 0$. This is Eq.~36 in the paper.

%----------------------------------------------------------------------
\subsection*{Boundary conditions}

In the paper we find ``At the bottom boundary $x=L$ we will assume that the system has no diffusive flux''.  We are not too sure what to make of this, but the author provides the following:\\\\
At $x=0$ (top of the domain):
\begin{eqnarray}
C_A(0,t)&=&C_{A}^0 \\
C_C(0,t)&=&C_{C}^0 \\
\hat{c}_k(0,t)&=&\hat{c}_{k}^0 \\
\phi(0,t) &=& \phi^0
\end{eqnarray}
At $x=L$ (bottom of the domain):
\[
\frac{\partial \hat{c}_k (x,t)}{\partial x}|_L = 0
\]
\[
\frac{\partial \phi (x,t)}{\partial x}|_L = 0
\]
As the paper is unclear, we employed two different strategies for the bottom BC.  The first was to follow the above strategy laid out in the paper and insist that the derivatives of the solutes and porosity (the variables with diffusion terms) are zero at the bottom boundary.  This was done by fixing the values in the final zone to be the same as those in the previous one, enforcing a flat derivative in that zone.  The alternative was to only force the diffusive derivative to zero, and to leave the advection term evaluated by a backward Euler derivative.  

%----------------------------------------------------------------------
\subsection*{Upwinding}
The equations for $C_A$ and $C_C$ are pure advection equations. We observed that in the absence of any stabilisation the computed fields show over- and undershoots, which are unphysical and which also find their way into the RHS of the other PDEs very rapidly due to the strong coupling. 
We therefore implemented an upwind scheme for these two equations, which switches between forward and backward differences dependant on the sign of the velocity:
\[
-U \frac{\partial C}{\partial x} \approx -U \left( \frac{C_{i+1} - C_{i-1}}{h} - S(U)\frac{C_{i+1} - 2C_{i} +C_{i-1}}{2h} \right)
\]
where $S(U)$ is the sign of $U$.  
\subsubsection*{Fiadeiro-Veronis scheme}
In \cite{lheu18} the Fiadeiro \& Veronis method \cite{five77,wrig92} is applied in the solute and porosity equations, for stability.  For the (steady state) advection-diffusion equation (see also p315 of \cite{boudreau}):
\[
\kappa \frac{T_{i+1}-2T_i + T_{i-1}}{h^2}
- u \frac{(1-\sigma) T_{i+1}+2\sigma T_i -(1+\sigma)T_{i-1}}{2 h} = 0
\]
which is a blend of backward (upstream) and central differences. The amount of blending is dictated by the value of the parameter $\sigma$, defined as
\[
\sigma 
= \text{coth} \frac{u h}{2 \kappa} - \frac{2 \kappa}{u h}
= \text{coth} \; Pe  - \frac{1}{Pe}
\]
The parameter $\sigma$ has the property that
$\sigma \rightarrow 0$ when $Pe \rightarrow 0$ and 
$\sigma \rightarrow 1$ when $Pe \rightarrow \infty$.
The equation above can also be rewritten:
\[
\kappa \frac{T_{i+1}-2T_i + T_{i-1}}{h^2}
- u \frac{1 T_{i+1}-T_{i-1}}{2 h} 
+ \frac{u \sigma h}{2} \frac{T_{i+1}-2 T_i +T_{i-1}}{h^2} = 0
\]
which makes the action of this stabilisation term more obvious: it is a diffusion term whose 
diffusion coefficient goes away when $h\rightarrow 0$.\\\\
Thus, if $\sigma = 0$ (diffusion dominated), then pure central differencing is obtained, and if $\sigma = 1$ (advection dominated), then pure backward differencing results. Interestingly enough, this blended or weighted scheme is second-order accurate even as it switches to backward differencing \cite{five77}.  When $\kappa \rightarrow 0$ (no physical diffusion at all - as is the case for $C_A$ and $C_C$) then $\sigma$ tends to $\pm 1$, and more precisely: $\sigma = sign(u)$, which becomes the simple upwinding scheme shown above.\\\\
We have also included an option for using this method in our code, although we note that it was not necessary for stability in our experience.

%----------------------------------

\subsubsection*{1st PDE}

\[
\frac{\partial {\color{teal} C_A}}{\partial t} 
= -U({\color{teal}\phi}) \frac{\partial {\color{teal}C_A}}{\partial x} 
- Da[(1-{\color{teal}C_A}){\color{teal}C_A}(\Omega_{DA}-\nu_1\Omega_{PA})+
\lambda {\color{teal}C_AC_C} (\Omega_{PC}-\nu_2\Omega_{DC})]
\]
Turning to the Aragonite PDE, we must now produce a FD approximation of the advection term.
For all the nodes inside the domain, we use an upwind scheme, which may be optionally turned off to return to a centered scheme:
\[
-U \frac{\partial C_A}{\partial x} \rightarrow -U_i \frac{C_A|_{i+1} -C_A|_{i-1}}{2h} + U_i S(U_i) \frac{C_A|_{i+1} - 2C_A|_{i} +C_A|_{i-1}}{2h}
\]
For the node at the top (i.e. $i=0$), we are prescribing the value of $C_A$, i.e.
the derivative is then zero so then 
\[
-U \frac{\partial C_A}{\partial x} \rightarrow 0 
\]
For the node at the bottom, no boundary condition is imposed so we use a backward
approximation:
\[
-U \frac{\partial C_A}{\partial x} \rightarrow -U_i \frac{C_A|_{i} -C_A|_{i-1}}{h}
\]
 
\begin{lstlisting}
def RHS_AR(self, AR, CA, c_ca, c_co, phi, x, h):
        dAR_dt = np.zeros(len(x))
        u = self.U(phi)
        
        for i in range(0,len(x)):
            # x = 0 BC, Dirichlet
            if (i==0):
                dAR_dt[i]= 0
            # x = Lx, no prescribed BC
            elif (i==len(x)-1):
                dAR_dt[i]= -u[i]*(AR[i]-AR[i-1])/h + self.R_AR(AR[i],CA[i],c_ca[i],c_co[i],x[i])
            else:
                dAR_dt[i]= -u[i]*( (AR[i+1]-AR[i-1])/(2*h) 
                        - self.upwind_switch*np.sign(u[i])*h/2*(AR[i+1] - 2*AR[i] + AR[i-1])/h**2 )\
                        + self.R_AR(AR[i],CA[i],c_ca[i],c_co[i], x[i])    
        return dAR_dt
\end{lstlisting}

%----------------------------------
\subsubsection*{2nd PDE}
\[
\frac{\partial {\color{teal}C_C}}{\partial t} 
= -U({\color{teal}\phi}) \frac{\partial {\color{teal}C_C}}{\partial x}  
+ Da[(1-{\color{teal}C_C}){\color{teal}C_C}(\Omega_{PC}-\nu_2\Omega_{DC})+
\lambda {\color{teal}C_AC_C} (\Omega_{DA}-\nu_1\Omega_{PA})]
\]

The structure of the 2nd PDE is identical to the first 
one so we adopt the same approach.

\begin{lstlisting}
def RHS_CA(self, AR, CA, c_ca, c_co, phi, x, h):
        dCA_dt = np.zeros(len(x))
        u = self.U(phi)
        for i in range(0,len(x)):
            # x = 0 BC, Dirichlet
            if (i==0):
                dCA_dt[i]= 0
            # x = Lx, no prescribed BC
            elif (i==len(x)-1):
                dCA_dt[i]= -u[i]*( CA[i]-CA[i-1] )/h + self.R_CA(AR[i],CA[i],c_ca[i],c_co[i],x[i])
            else:
                dCA_dt[i]= -u[i]*( ( CA[i+1]-CA[i-1])/(2*h) 
                    - self.upwind_switch*np.sign(u[i])*h/2*(CA[i+1] - 2*CA[i] + CA[i-1])/h**2 )\
                    + self.R_CA(AR[i],CA[i],c_ca[i],c_co[i],x[i])
        return dCA_dt
\end{lstlisting}

%----------------------------------
\subsubsection*{3rd \& 4th PDE}

\[
\frac{\partial {\color{teal}\hat{c}_k}}{\partial t} 
= -W({\color{teal}\phi}) \frac{\partial {\color{teal}\hat{c}_k}}{\partial x}
+\frac{1}{\color{teal}\phi} \frac{\partial}{\partial x} 
\left( {\color{teal}\phi} d_k \frac{\partial {\color{teal}\hat{c}_k}}{\partial x} \right)
+Da \frac{1-{\color{teal}\phi}}{{\color{teal}\phi}}(\delta-{\color{teal}\hat{c}_k})
[{\color{teal}C_A}(\Omega_{DA}-\nu_1\Omega_{PA})-\lambda 
{\color{teal}C_C} (\Omega_{PC}-\nu_2 \Omega_{DC})  ]
\]
The advection term is treated as above however instead of a simple upwind scheme we here use the Fiadeiro-Veronis scheme.  We also have to discretise a diffusion term with a non constant diffusion coefficient.  The standard procedure in such cases is as follows:
\[
\frac{\partial }{\partial x} \left( k \frac{\partial f}{\partial x} \right)
\simeq
\frac{
k_{i+1/2} \frac{f_{i+1}-f_i}{h}
-
k_{i-1/2} \frac{f_{i}-f_{i-1}}{h}
}{h}
\] 
where $k_{i\pm 1/2}$ is evaluated between the points to maintain the 
second order accuracy.  In our case we then have (with the $k$ subscript removed):
\[
\frac{\partial}{\partial x} 
\left( {\color{teal}\phi} d 
\frac{\partial {\color{teal}\hat{c}}}{\partial x} \right)
\simeq 
\frac{
\phi_{i+1/2} d_{i+1/2} \frac{c_{i+1}-c_i}{h}
-
\phi_{i-1/2} d_{i-1/2} \frac{c_{i}-c_{i-1}}{h}
}{h}
\]
The value of $c$ is prescribed at $x=0$ but only its derivative
is prescribed at the bottom ($x=L$):
\[
\frac{\partial \hat{c} (x,t)}{\partial x}|_L = 0
\]
i.e.
\[
\frac{c_i-c_{i-1}}{h}=0  \qquad \text{for} \quad i=nnx-1
\] 
We then write the expression above fully backward:
\[
\frac{\partial}{\partial x} 
\left( {\color{teal}\phi} d 
\frac{\partial {\color{teal}\hat{c}}}{\partial x} \right)
\simeq 
\frac{
\phi_{i-1/2} d_{i-1/2} \frac{c_{i}-c_{i-1}}{h}
-
\phi_{i-3/2} d_{i-3/2} \frac{c_{i-1}-c_{i-2}}{h}
}{h}
\]
and we see that the b.c. kills the first term, so we are left with
\[
\frac{\partial}{\partial x} 
\left( {\color{teal}\phi} d 
\frac{\partial {\color{teal}\hat{c}}}{\partial x} \right)
\simeq 
\frac{-\phi_{i-3/2} d_{i-3/2} \frac{c_{i-1}-c_{i-2}}{h}}{h}
=
-\phi_{i-3/2} d_{i-3/2} \frac{c_{i-1}-c_{i-2}}{h^2}
\]

\begin{lstlisting}
def RHS_c_ca(self, AR, CA, c_ca, c_co, phi, x, h):
        dc_ca_dt = np.zeros(len(x))
        w = self.W(phi)
        phi_half = np.zeros(len(x)-1)
        d_ca_half = np.zeros(len(x)-1)
        
        for i in range(0,len(x)-1):
            phi_half[i] = ( phi[i+1] + phi[i] ) / 2
            d_ca_half[i] = ( self.d_c_ca(phi[i+1]) + self.d_c_ca(phi[i]) ) / 2
        
        for i in range(0,len(x)):
            # x = 0 BC, Dirichlet
            if (i==0):
                dc_ca_dt[i] = 0
            # x = Lx BC, df/dx = 0
            elif (i==len(x)-1):
                dc_ca_dt[i] =  ( 1 / phi[i] )*( phi[i-2] * self.d_c_ca(phi[i-2]) * c_ca[i-2] -\
                                                phi[i-1] * self.d_c_ca(phi[i-1]) * c_ca[i-1])/h**2\  
                                + self.R_c_ca(AR[i], CA[i], c_ca[i], c_co[i], phi[i], x[i]) 
            else:
                dc_ca_dt[i] = - w[i]*( (c_ca[i+1] - c_ca[i-1]) / (2*h) +\
                                    -self.FV_switch*self.sigma_ca(w[i], h, phi[i])\
                                    *h/2*(c_ca[i+1] - 2*c_ca[i] + c_ca[i-1])/(h**2) )\
                              + ( 1 / phi[i] ) * (phi_half[i]*d_ca_half[i]*(c_ca[i+1] - c_ca[i]) -\
                                        phi_half[i-1]*d_ca_half[i-1]*(c_ca[i] - c_ca[i-1]))/h**2\
                               + self.R_c_ca(AR[i], CA[i], c_ca[i], c_co[i], phi[i], x[i])
        return dc_ca_dt
\end{lstlisting}

%----------------------------------
\subsubsection*{5th PDE}
\[
\frac{\partial {\color{teal}\phi}}{\partial t} 
= -\frac{\partial}{\partial x} (W {\color{teal}\phi})
+d_{\phi} \frac{\partial^2 {\color{teal}\phi}}{\partial x^2} + Da (1-{\color{teal}\phi})
[{\color{teal}C_A}(\Omega_{DA}-\nu_1 \Omega_{PA})-\lambda 
{\color{teal}C_C} (\Omega_{PC}-\nu_2 \Omega_{DC})] \nn
\]
The advection term in this case is of a slightly different form, so we apply the Fiadeiro-Veronis scheme as follows:
\[
\frac{\partial}{\partial x} (W ({\color{teal}\phi}) {\color{teal}\phi})
\simeq
\frac{W(\phi_{i+1})\phi_{i+1} - W(\phi_{i-1})\phi_{i-1}}{2h} - \sigma_i \frac{\phi_{i-1}W(\phi_{i-1}) - 2 \phi_iW(\phi_i) + \phi_{i-1} W(\phi_{i-1})}{2h}
\]
The diffusion derivative in this case is just a standard centered difference
\[
\frac{\partial^2 {\color{teal}\phi}}{\partial x^2}
\simeq \frac{\phi_{i+1}-\phi_{i-1}}{2h}  
\]
The value of $\phi$ is prescribed at $x=0$ but only its derivative
is prescribed at the bottom ($x=L$):
\[
\frac{\partial \phi (x,t)}{\partial x}|_L = 0
\]
i.e.
\[
\frac{\phi_i-\phi_{i-1}}{h}=0  \qquad \text{for} \quad i=nnx-1
\] 
The fully backward diffusion term on the right side is 
\[
\frac{\phi_{i-2}-2\phi_{i-1}+\phi_i}{h^2}
\]
which means that given the boundary condition it becomes
\[
\frac{\phi_{i-2}-\phi_{i-1}}{h^2}
\]
In the code this is implemented as 
\begin{lstlisting}
def RHS_phi(self, AR, CA, c_ca, c_co, phi, x, h):
        dphi_dt = np.zeros(len(x))
        w = self.W(phi)
        for i in range(0,len(x)):
            # x = 0 BC, Dirichlet
            if (i==0):
                dphi_dt[i] = 0
            # x = Lx BC, df/dx = 0
            elif (i==len(x)-1):
                dphi_dt[i] = - ( w[i] * phi[i] - w[i-1] * phi[i-1] ) / h +\
                             self.d_phi*( phi[i-2] - phi[i-1] ) / h**2 +\
                             self.R_phi(AR[i], CA[i], c_ca[i], c_co[i], phi[i], x[i])
            else:
                dphi_dt[i] = -( w[i+1] * phi[i+1] - w[i-1] * phi[i-1] ) / (2*h) +\
                    self.FV_switch*self.sigma_phi(w[i], h)*h/2*(phi[i+1]*w[i+1] - 2*phi[i]*w[i]                                                                 + phi[i-1]*w[i-1])/(h**2) +\
                            self.d_phi * ( phi[i+1] - 2*phi[i] + phi[i-1] ) / h**2 +\
                            self.R_phi(AR[i], CA[i], c_ca[i], c_co[i], phi[i], x[i])
        return dphi_dt
\end{lstlisting}

\printbibliography

\end{document}
